\input{Style/style}

\begin{document}


% \title{标题设置见Style/style.tex中的标题样式}
% \author{}

% \input{Content/titlepage.tex} % 这个是封面,不需要刻意注释掉
% \maketitle

% \begin{abstract}
% 	摘要。
% 	\\\\
% 	\textbf{关键词:}关键词1;关键词2
% \end{abstract}

\section{DDPM 的理论与实践}
\subsection{DDPM 的工作流}



\subsection{关键环节:如何实现从 \(t\) 到 \(t-1\) 的飞跃?}

DDPM 假设前向过程是固定的马尔可夫链:
\[
q(x_t | x_{t-1}) = \mathcal{N}(x_t; \sqrt{\alpha_t} x_{t-1}, \beta_t I)
\]
其中 \(\alpha_t = 1 - \beta_t\),\(\bar{\alpha}_t = \prod_{s=1}^t \alpha_s\).


通过贝叶斯规则和高斯条件分布的性质,可以得到:
\[
q(x_{t-1} | x_t, x_0) = \mathcal{N}(x_{t-1}; \tilde{\mu}(x_t, x_0), \tilde{\beta}_t I)
\]

其中\textbf{真实均值}为:
\begin{equation}
\tilde{\mu}(x_t, x_0) = \frac{\sqrt{\bar{\alpha}_{t-1}} \beta_t}{1 - \bar{\alpha}_t} x_0 + \frac{\sqrt{\alpha_t} (1 - \bar{\alpha}_{t-1})}{1 - \bar{\alpha}_t} x_t
\label{mu}
\end{equation}

注意,这个均值仅为 \(x_0\) 与 \(x_t\) 的函数,其中 \(x_t\) 就是当前的带噪图像. 故问题仅是对 \(x_0\) 做估计,并用估计值来替代公式中的实际值. 此部分不在本节赘述. 

以下详细展示式~\eqref{mu}的推导过程:


\begin{proof}

已知:(1)~前向过程是高斯马尔可夫链;(2)~\(x_t\) 和 \(x_{t-1}\) 都与 \(x_0\) 有线性高斯关系;
求:后验分布 \( q(x_{t-1} \mid x_t, x_0) \) 的均值 \(\tilde{\mu}(x_t, x_0)\).

从 DDPM 的前向定义出发,写出三个基本的前向过程:

(a) \(x_t\) 给定 \(x_0\):
\[
x_t = \sqrt{\bar{\alpha}_t} x_0 + \sqrt{1 - \bar{\alpha}_t} \epsilon_t,\quad \epsilon_t \sim \mathcal{N}(0, I)
\]

\(\Rightarrow\)
\[
q(x_t \mid x_0) = \mathcal{N}(x_t; \sqrt{\bar{\alpha}_t} x_0,\ (1 - \bar{\alpha}_t) I)
\tag{A}
\]

(b) \(x_{t-1}\) 给定 \(x_0\):
\[
q(x_{t-1} \mid x_0) = \mathcal{N}(x_{t-1}; \sqrt{\bar{\alpha}_{t-1}} x_0,\ (1 - \bar{\alpha}_{t-1}) I)
\tag{B}
\]

(c) \(x_t\) 给定 \(x_{t-1}\)(马尔可夫性):
\[
q(x_t \mid x_{t-1}) = \mathcal{N}(x_t; \sqrt{\alpha_t} x_{t-1},\ \beta_t I)
\tag{C}
\]

我们想求 \( q(x_{t-1} \mid x_t, x_0) \)。由于前向过程是马尔可夫的,且所有关系都是高斯的,这个后验也是高斯的。

回顾贝叶斯规则:
\begin{tcolorbox}
\[
p(x \mid y) = \frac{p(xy)}{p(y)} = \frac{p(y \mid x) \cdot p(x)}{p(y)}
\]
\end{tcolorbox}

据此有:
\[
q(x_{t-1} \mid x_t, x_0) = \frac{q(x_t|x_{t-1},x_0)\cdot q(x_{t-1}|x_0)}{q(x_t|x_0)}
\]

由于 \(x_t\)  和 \(x_0\)是已知的观测值或条件,而我们要求的是关于 \(x_{t-1}\) 的分布,分母 \(q(x_t|x_0)\) 对于变量 \(x_{t-1}\)   来说就是一个常数。故将其约去,进一步有:
\[
q(x_{t-1} \mid x_t, x_0)  \propto q(x_t \mid x_{t-1}, x_0) \cdot q(x_{t-1} \mid x_0)
\]

因为马尔可夫性:\(
q(x_t \mid x_{t-1}, x_0) = q(x_t \mid x_{t-1})
\),所以:
\[
q(x_{t-1} \mid x_t, x_0)  \propto q(x_t \mid x_{t-1}) \cdot q(x_{t-1} \mid x_0)
\]

即后验\(\propto\)似然\(\times\)先验,其中:

似然:\( q(x_t \mid x_{t-1}) = \mathcal{N}(x_t; \sqrt{\alpha_t} x_{t-1}, \beta_t I) \)

先验:\( q(x_{t-1} \mid x_0) = \mathcal{N}(x_{t-1}; \sqrt{\bar{\alpha}_{t-1}} x_0, (1 - \bar{\alpha}_{t-1}) I) \)

接下来的步骤涉及联合高斯条件均值,先引入一个结论:

\begin{tcolorbox}
对于联合高斯变量 
\[\begin{bmatrix} y \\ x \end{bmatrix} \sim \mathcal{N}\left( \begin{bmatrix} \mu_y \\ \mu_x \end{bmatrix}, \begin{bmatrix} \Sigma_{yy} & \Sigma_{yx} \\ \Sigma_{xy} & \Sigma_{xx} \end{bmatrix} \right),\]  
其条件均值为:
\[
\mathbb{E}[y \mid x] = \mu_y + \Sigma_{yx} \Sigma_{xx}^{-1} (x - \mu_x)
\]

证明参见~\ref{多元高斯分布的条件均值}.
\end{tcolorbox}

回到我们的情况,注意到 \(x_{t-1}\) 和 \(x_t\) 在给定 \(x_0\) 下是联合高斯的:
\[
\begin{bmatrix} x_{t-1} \\ x_t \end{bmatrix} \sim \mathcal{N}\left( 
\begin{bmatrix} \sqrt{\bar{\alpha}_{t-1}} x_0 \\ \sqrt{\bar{\alpha}_t} x_0 \end{bmatrix},
\begin{bmatrix}
1 - \bar{\alpha}_{t-1} & \text{Cov}(x_{t-1}, x_t) \\
\text{Cov}(x_t, x_{t-1}) & 1 - \bar{\alpha}_t
\end{bmatrix} 
\right)
\]

计算协方差,注意到\(x_t = \sqrt{\alpha_t} x_{t-1} + \sqrt{\beta_t} \epsilon\),且回顾(B) 式已有\(\text{Var}(x_{t-1}) = (1 - \bar{\alpha}_{t-1})I\),即容易得:
\[
\text{Cov}(x_{t-1}, x_t) = \text{Cov}(x_{t-1},  \sqrt{\alpha_t} x_{t-1} + \sqrt{\beta_t} \epsilon) =  \sqrt{\alpha_t} \text{Var}(x_{t-1}) = \sqrt{\alpha_t} (1 - \bar{\alpha}_{t-1})
\]

故
\[
\mathbb{E}[x_{t-1} \mid x_t, x_0] = \sqrt{\bar{\alpha}_{t-1}} x_0 + \sqrt{\alpha_t} (1 - \bar{\alpha}_{t-1}) \cdot  (1 - \bar{\alpha}_t) ^{-1} (x_t - \sqrt{\bar{\alpha}_t} x_0)
\]

注意到,以上的均值仅与 \(x_0\) 和 \(x_t\) 有关,故写为函数形式并整理,得到:
\[
\tilde{\mu}(x_t, x_0) =  \left[ \sqrt{\bar{\alpha}_{t-1}} - \frac{ \sqrt{\alpha_t} (1 - \bar{\alpha}_{t-1}) \sqrt{\bar{\alpha}_t} }{1 - \bar{\alpha}_t} \right] x_0 + \frac{ \sqrt{\alpha_t} (1 - \bar{\alpha}_{t-1}) }{1 - \bar{\alpha}_t} x_t
\]

注意到\(\bar{\alpha}_t = \bar{\alpha}_{t-1} \alpha_t\),所以 \(\sqrt{\bar{\alpha}_t} = \sqrt{\bar{\alpha}_{t-1}} \sqrt{\alpha_t}\),进一步整理 \(x_0\) 的系数:

\[
 \sqrt{\bar{\alpha}_{t-1}} - \frac{ \sqrt{\alpha_t} (1 - \bar{\alpha}_{t-1}) \sqrt{\bar{\alpha}_t} }{1 - \bar{\alpha}_t} =
 \sqrt{\bar{\alpha}_{t-1}} \left[ 1 - \frac{ \alpha_t (1 - \bar{\alpha}_{t-1}) }{1 - \bar{\alpha}_t} \right]
= \sqrt{\bar{\alpha}_{t-1}} \cdot \frac{ (1 - \bar{\alpha}_t) - \alpha_t (1 - \bar{\alpha}_{t-1}) }{1 - \bar{\alpha}_t}=\frac{\sqrt{\bar{\alpha}_{t-1}} \beta_t}{1 - \bar{\alpha}_t}
\]

最终,
\[
\tilde{\mu}(x_t, x_0) = \frac{ \sqrt{\bar{\alpha}_{t-1}} \beta_t }{1 - \bar{\alpha}_t } x_0 + \frac{ \sqrt{\alpha_t} (1 - \bar{\alpha}_{t-1}) }{1 - \bar{\alpha}_t } x_t
\]


\end{proof}




\subsection{回到实践中:从代码运行逻辑的视角看 DDPM 到底发生了什么}



\section{附录}

\subsection{多元高斯分布的条件均值}
\label{多元高斯分布的条件均值}

\begin{conclusion}
对于联合高斯变量  
\[
\begin{bmatrix} y \\ x \end{bmatrix} \sim \mathcal{N}\left( 
\begin{bmatrix} \mu_y \\ \mu_x \end{bmatrix},
\begin{bmatrix}
\Sigma_{yy} & \Sigma_{yx} \\
\Sigma_{xy} & \Sigma_{xx}
\end{bmatrix}
\right),
\]  
其条件分布 \( p(y \mid x) \) 也是高斯分布,且  
\[
\mathbb{E}[y \mid x] = \mu_y + \Sigma_{yx} \Sigma_{xx}^{-1} (x - \mu_x)
\]
\end{conclusion}

这是多元高斯分布(Multivariate Gaussian)的一个经典结果,我们先在二元的情形下考察它:


\printbibliography%[heading=bibliography,title=参考文献]
\end{document}