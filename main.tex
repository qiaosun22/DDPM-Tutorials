\input{Style/style}

\begin{document}


% \title{标题设置见Style/style.tex中的标题样式}
% \author{}

% \input{Content/titlepage.tex} % 这个是封面,不需要刻意注释掉
% \maketitle

% \begin{abstract}
% 	摘要. 
% 	\\\\
% 	\textbf{关键词:}关键词1;关键词2
% \end{abstract}

\section{DDPM 的理论与实践}
\subsection{DDPM 的工作流}

DDPM 是经典的高斯过程,虽然前向过程中的加噪可以\textbf{被巧妙地设计为一次计算就加到第 \(t\) 步},但是反向过程中务必要逐步(step)地从后向前去噪. 每次去噪步 \(t\) 上都是已知带噪数据 \(x_t\) 求该步去噪后得到的噪声更少的数据  \(x_{t-1}\). 

但是,要厘清一个理解上的误区,即认为“去噪过程是每步预测当前步的噪声\(\epsilon_t\)”. 这其实是不可能的,因为前向过程既然已经“被巧妙地设计为一次计算就加到第 \(t\) 步”,那么这之前的\(\{1, ..., t-1\}\)步中任何一步的噪声在构造训练数据过程中都是没有的,也就无从去预测. 事实上,我们预测的只能是加噪时“一次计算就加到第 \(t\) 步的噪声\(\epsilon_{1:t}\)”. 

但是这里一个令人困惑的问题就来了,既然这样,为什么我们不直接从噪声中去掉它来得到干净的数据呢?为什么还要“舍近求远”地去把这个预测的累积 \(t\) 步的噪声 \(\epsilon_{1:t}\) 和当前数据 \(x_t\) 步之间做差值来得到仅仅向前去噪一步(相当于去掉 \(\epsilon_{t}\) )的结果?

事实上,你完全可以!DDIM 和 Flow Matching 就是这么做的(用少步甚至 1 步生成)。但目前我们讨论的是 DDPM,它选择逐步去噪的策略可以被归为历史的局限。而由 DDPM 到 DDIM 再到 Flow Matching 的演变其实恰恰反映了扩散模型发展过程中认识由浅入深的过程:

(a) 历史原因:DDPM 基于变分推断框架
它假设反向过程是一个马尔可夫链,每步都是高斯分布;
为了最大化证据下界(ELBO),需要逐步建模 \(q(x_{t-1},x_t)\);
所以即使能一步到位,训练目标仍被设计为逐步去噪。
(b) 稳定性考虑(早期认知)
在 DDPM 提出时(2020),人们认为多步去噪更稳定;
一步去噪对模型误差敏感(若 \(\epsilon_\theta\) 有偏差, \(x_0\) 会严重失真);
逐步去噪可“逐步修正”误差。
但后来 DDIM(2020)和 Flow Matching(2023)证明:只要采样策略得当,少步甚至 1 步也可以高质量生成。



% 模型预测初始噪声\(\epsilon\),


\subsection{关键环节:如何实现从 \(t\) 到 \(t-1\) 的飞跃?}

DDPM 假设前向过程是固定的马尔可夫链:
\[
q(x_t | x_{t-1}) = \mathcal{N}(x_t; \sqrt{\alpha_t} x_{t-1}, \beta_t I)
\]
其中 \(\alpha_t = 1 - \beta_t\),\(\bar{\alpha}_t = \prod_{s=1}^t \alpha_s\).


通过贝叶斯规则和高斯条件分布的性质,可以得到:
\[
q(x_{t-1} | x_t, x_0) = \mathcal{N}(x_{t-1}; \tilde{\mu}(x_t, x_0), \tilde{\beta}_t I)
\]

其中\textbf{真实均值}为:
\begin{equation}
\tilde{\mu}(x_t, x_0) = \frac{\sqrt{\bar{\alpha}_{t-1}} \beta_t}{1 - \bar{\alpha}_t} x_0 + \frac{\sqrt{\alpha_t} (1 - \bar{\alpha}_{t-1})}{1 - \bar{\alpha}_t} x_t
\label{mu}
\end{equation}

注意,这个均值仅为 \(x_0\) 与 \(x_t\) 的函数,其中 \(x_t\) 就是当前的带噪图像. 故问题仅是对 \(x_0\) 做估计,并用估计值来替代公式中的实际值. 此部分不在本节赘述. 

以下详细展示式~\eqref{mu}的推导过程:


\begin{proof}

已知:(1)~前向过程是高斯马尔可夫链;(2)~\(x_t\) 和 \(x_{t-1}\) 都与 \(x_0\) 有线性高斯关系;
求:后验分布 \( q(x_{t-1} \mid x_t, x_0) \) 的均值 \(\tilde{\mu}(x_t, x_0)\).

从 DDPM 的前向定义出发,写出三个基本的前向过程:

(a) \(x_t\) 给定 \(x_0\):
\[
x_t = \sqrt{\bar{\alpha}_t} x_0 + \sqrt{1 - \bar{\alpha}_t} \epsilon_t,\quad \epsilon_t \sim \mathcal{N}(0, I)
\]

\(\Rightarrow\)
\[
q(x_t \mid x_0) = \mathcal{N}(x_t; \sqrt{\bar{\alpha}_t} x_0,\ (1 - \bar{\alpha}_t) I)
\tag{A}
\]

(b) \(x_{t-1}\) 给定 \(x_0\):
\[
q(x_{t-1} \mid x_0) = \mathcal{N}(x_{t-1}; \sqrt{\bar{\alpha}_{t-1}} x_0,\ (1 - \bar{\alpha}_{t-1}) I)
\tag{B}
\]

(c) \(x_t\) 给定 \(x_{t-1}\)(马尔可夫性):
\[
q(x_t \mid x_{t-1}) = \mathcal{N}(x_t; \sqrt{\alpha_t} x_{t-1},\ \beta_t I)
\tag{C}
\]

我们想求 \( q(x_{t-1} \mid x_t, x_0) \). 由于前向过程是马尔可夫的,且所有关系都是高斯的,这个后验也是高斯的. 

回顾贝叶斯规则:
\begin{tcolorbox}
\[
p(x \mid y) = \frac{p(xy)}{p(y)} = \frac{p(y \mid x) \cdot p(x)}{p(y)}
\]
\end{tcolorbox}

据此有:
\[
q(x_{t-1} \mid x_t, x_0) = \frac{q(x_t|x_{t-1},x_0)\cdot q(x_{t-1}|x_0)}{q(x_t|x_0)}
\]

由于 \(x_t\)  和 \(x_0\)是已知的观测值或条件,而我们要求的是关于 \(x_{t-1}\) 的分布,分母 \(q(x_t|x_0)\) 对于变量 \(x_{t-1}\)   来说就是一个常数. 故将其约去,进一步有:
\[
q(x_{t-1} \mid x_t, x_0)  \propto q(x_t \mid x_{t-1}, x_0) \cdot q(x_{t-1} \mid x_0)
\]

因为马尔可夫性:\(
q(x_t \mid x_{t-1}, x_0) = q(x_t \mid x_{t-1})
\),所以:
\[
q(x_{t-1} \mid x_t, x_0)  \propto q(x_t \mid x_{t-1}) \cdot q(x_{t-1} \mid x_0)
\]

即后验\(\propto\)似然\(\times\)先验,其中:

似然:\( q(x_t \mid x_{t-1}) = \mathcal{N}(x_t; \sqrt{\alpha_t} x_{t-1}, \beta_t I) \)

先验:\( q(x_{t-1} \mid x_0) = \mathcal{N}(x_{t-1}; \sqrt{\bar{\alpha}_{t-1}} x_0, (1 - \bar{\alpha}_{t-1}) I) \)

接下来的步骤涉及联合高斯条件均值,先引入一个结论:

\begin{tcolorbox}
对于联合高斯变量 
\[\begin{bmatrix} y \\ x \end{bmatrix} \sim \mathcal{N}\left( \begin{bmatrix} \mu_y \\ \mu_x \end{bmatrix}, \begin{bmatrix} \Sigma_{yy} & \Sigma_{yx} \\ \Sigma_{xy} & \Sigma_{xx} \end{bmatrix} \right),\]  
其条件均值为:
\[
\mathbb{E}[y \mid x] = \mu_y + \Sigma_{yx} \Sigma_{xx}^{-1} (x - \mu_x)
\]

证明参见~\ref{多元高斯分布的条件均值}.
\end{tcolorbox}

回到我们的情况,注意到 \(x_{t-1}\) 和 \(x_t\) 在给定 \(x_0\) 下是联合高斯的:
\[
\begin{bmatrix} x_{t-1} \\ x_t \end{bmatrix} \sim \mathcal{N}\left( 
\begin{bmatrix} \sqrt{\bar{\alpha}_{t-1}} x_0 \\ \sqrt{\bar{\alpha}_t} x_0 \end{bmatrix},
\begin{bmatrix}
1 - \bar{\alpha}_{t-1} & \text{Cov}(x_{t-1}, x_t) \\
\text{Cov}(x_t, x_{t-1}) & 1 - \bar{\alpha}_t
\end{bmatrix} 
\right)
\]

计算协方差,注意到\(x_t = \sqrt{\alpha_t} x_{t-1} + \sqrt{\beta_t} \epsilon\),且回顾(B) 式已有\(\text{Var}(x_{t-1}) = (1 - \bar{\alpha}_{t-1})I\),即容易得:
\[
\text{Cov}(x_{t-1}, x_t) = \text{Cov}(x_{t-1},  \sqrt{\alpha_t} x_{t-1} + \sqrt{\beta_t} \epsilon) =  \sqrt{\alpha_t} \text{Var}(x_{t-1}) = \sqrt{\alpha_t} (1 - \bar{\alpha}_{t-1})
\]

故
\[
\mathbb{E}[x_{t-1} \mid x_t, x_0] = \sqrt{\bar{\alpha}_{t-1}} x_0 + \sqrt{\alpha_t} (1 - \bar{\alpha}_{t-1}) \cdot  (1 - \bar{\alpha}_t) ^{-1} (x_t - \sqrt{\bar{\alpha}_t} x_0)
\]

注意到,以上的均值仅与 \(x_0\) 和 \(x_t\) 有关,故写为函数形式并整理,得到:
\[
\tilde{\mu}(x_t, x_0) =  \left[ \sqrt{\bar{\alpha}_{t-1}} - \frac{ \sqrt{\alpha_t} (1 - \bar{\alpha}_{t-1}) \sqrt{\bar{\alpha}_t} }{1 - \bar{\alpha}_t} \right] x_0 + \frac{ \sqrt{\alpha_t} (1 - \bar{\alpha}_{t-1}) }{1 - \bar{\alpha}_t} x_t
\]

注意到\(\bar{\alpha}_t = \bar{\alpha}_{t-1} \alpha_t\),所以 \(\sqrt{\bar{\alpha}_t} = \sqrt{\bar{\alpha}_{t-1}} \sqrt{\alpha_t}\),进一步整理 \(x_0\) 的系数:

\[
 \sqrt{\bar{\alpha}_{t-1}} - \frac{ \sqrt{\alpha_t} (1 - \bar{\alpha}_{t-1}) \sqrt{\bar{\alpha}_t} }{1 - \bar{\alpha}_t} =
 \sqrt{\bar{\alpha}_{t-1}} \left[ 1 - \frac{ \alpha_t (1 - \bar{\alpha}_{t-1}) }{1 - \bar{\alpha}_t} \right]
= \sqrt{\bar{\alpha}_{t-1}} \cdot \frac{ (1 - \bar{\alpha}_t) - \alpha_t (1 - \bar{\alpha}_{t-1}) }{1 - \bar{\alpha}_t}=\frac{\sqrt{\bar{\alpha}_{t-1}} \beta_t}{1 - \bar{\alpha}_t}
\]

最终,
\[
\tilde{\mu}(x_t, x_0) = \frac{ \sqrt{\bar{\alpha}_{t-1}} \beta_t }{1 - \bar{\alpha}_t } x_0 + \frac{ \sqrt{\alpha_t} (1 - \bar{\alpha}_{t-1}) }{1 - \bar{\alpha}_t } x_t
\]

\end{proof}




\subsection{回到实践中:从代码运行逻辑的视角看 DDPM 到底发生了什么}

模型预测的是 \(\epsilon\),但这并不意味着去噪过程直接“减去 \(\epsilon\)”。实际上,去噪的每一步都由调度器(scheduler),而神经网络仅负责提供对原始噪声的估计。整个生成流程可分解为以下三个阶段:

1. 噪声预测(模型部分)  
   给定当前带噪样本 \(x_t\) 和时间步 \(t\),神经网络输出:
   \[
   \epsilon_\theta = \texttt{model}(x_t, t)
   \]
   这是唯一由可学习参数参与的步骤。

2. 干净图像重建(确定性计算)  
   利用预测的 \(\epsilon_\theta\),通过重参数化公式反推对原始数据的估计:
   \[
   \hat{x}_0 = \frac{x_t - \sqrt{1 - \bar{\alpha}_t} \cdot \epsilon_\theta}{\sqrt{\bar{\alpha}_t}}
   \]
   此步骤完全由预设的噪声调度 \(\{\bar{\alpha}_t\}\) 决定,无需学习。

3. 下一步状态计算(调度器核心)  
   将 \(\hat{x}_0\) 代入理论均值公式~\eqref{mu},得到去噪后的均值:
   \[
   \mu_\theta = \tilde{\mu}(x_t, \hat{x}_0) = \frac{ \sqrt{\bar{\alpha}_{t-1}} \beta_t }{1 - \bar{\alpha}_t } \hat{x}_0 + \frac{ \sqrt{\alpha_t} (1 - \bar{\alpha}_{t-1}) }{1 - \bar{\alpha}_t } x_t
   \]
   在 DDPM 中,最终的 \(x_{t-1}\) 为:
   \[
   x_{t-1} = \mu_\theta + \sigma_t \cdot z,\quad z \sim \mathcal{N}(0, I)
   \]
   其中方差 \(\sigma_t\) 由调度器设定(如 \(\sigma_t = \beta_t\) 或 \(\tilde{\beta}_t\)),而随机项 \(z\) 引入探索性噪声,使采样过程保持随机性。

\begin{tcolorbox}
    关键分工:  \\
    模型(Model):仅预测 \(\epsilon\);  \\
    调度器(Scheduler):负责所有与时间步相关的计算(\(\bar{\alpha}_t, \beta_t\))、\(\hat{x}_0\) 重建、\(\mu_\theta\) 计算、以及是否添加随机噪声。  
\end{tcolorbox}


这种解耦设计使得同一模型可配合不同调度策略(如 DDPM、DDIM、DPM-Solver)使用,极大提升了灵活性。例如,在 DDIM 中,调度器仅需将 \(\sigma_t\) 设为 0,即可实现确定性跳步采样,而模型本身无需任何修改。

因此,DDPM 的“逐步去噪”并非源于模型能力的限制,而是其默认调度策略的选择。一旦更换调度器,同一模型即可实现快速、确定性生成——这正是从 DDPM 到 DDIM 再到 Flow Matching 的演进所揭示的深层洞见:生成过程的效率瓶颈不在模型,而在采样策略。



\section{附录}

\subsection{多元高斯分布的条件均值}
\label{多元高斯分布的条件均值}

\begin{conclusion}
对于联合高斯变量  
\[
\begin{bmatrix} y \\ x \end{bmatrix} \sim \mathcal{N}\left( 
\begin{bmatrix} \mu_y \\ \mu_x \end{bmatrix},
\begin{bmatrix}
\Sigma_{yy} & \Sigma_{yx} \\
\Sigma_{xy} & \Sigma_{xx}
\end{bmatrix}
\right),
\]  
其条件分布 \( p(y \mid x) \) 也是高斯分布,且  
\[
\mathbb{E}[y \mid x] = \mu_y + \Sigma_{yx} \Sigma_{xx}^{-1} (x - \mu_x)
\]
\end{conclusion}

这是多元高斯分布(Multivariate Gaussian)的一个经典结果,我们先在二元的情形下考察它:

\begin{proof}
    
步骤 1:写出联合概率密度函数

联合高斯密度为:
\[
p(y, x) = \frac{1}{(2\pi)^{(n+m)/2} |\Sigma|^{1/2}} \exp\left( -\frac{1}{2} \begin{bmatrix} y - \mu_y \\ x - \mu_x \end{bmatrix}^\top \Sigma^{-1} \begin{bmatrix} y - \mu_y \\ x - \mu_x \end{bmatrix} \right)
\]

其中 \(\Sigma = \begin{bmatrix} \Sigma_{yy} & \Sigma_{yx} \\ \Sigma_{xy} & \Sigma_{xx} \end{bmatrix}\),且 \(\Sigma_{xy} = \Sigma_{yx}^\top\)(协方差矩阵对称). 


步骤 2:将二次型展开(配方法)

令:
\[
\Delta_y = y - \mu_y,\quad \Delta_x = x - \mu_x
\]

则指数部分为:
\[
Q = \begin{bmatrix} \Delta_y \\ \Delta_x \end{bmatrix}^\top \Sigma^{-1} \begin{bmatrix} \Delta_y \\ \Delta_x \end{bmatrix}
\]

我们不直接求 \(\Sigma^{-1}\),而是用\textbf{矩阵分块求逆}或\textbf{配方法}(completing the square). 

\begin{tcolorbox}
关键思想:把 \(Q\) 写成关于 \(\Delta_y\) 的二次函数:
\[
Q = (\Delta_y - A \Delta_x)^\top M (\Delta_y - A \Delta_x) + \text{terms only in } \Delta_x
\]
这样,条件分布 \(p(y \mid x)\) 的均值就是 \(\mu_y + A \Delta_x\).

\end{tcolorbox}

步骤 3:使用矩阵恒等式(标准推导)

已知协方差矩阵的分块形式,其逆矩阵可表示为(利用 Schur complement):

\[
\Sigma^{-1} = 
\begin{bmatrix}
\Sigma_{yy}^{-1} + \Sigma_{yy}^{-1} \Sigma_{yx} S^{-1} \Sigma_{xy} \Sigma_{yy}^{-1} & -\Sigma_{yy}^{-1} \Sigma_{yx} S^{-1} \\
- S^{-1} \Sigma_{xy} \Sigma_{yy}^{-1} & S^{-1}
\end{bmatrix}
\]
其中 \(S = \Sigma_{xx} - \Sigma_{xy} \Sigma_{yy}^{-1} \Sigma_{yx}\) 是 Schur complement. 

但更简单的方式是直接假设条件均值为线性形式(高斯分布的性质保证它是线性的):

设:
\[
\mathbb{E}[y \mid x] = \mu_y + K (x - \mu_x)
\]
我们的目标是求矩阵 \(K\). 


经典推导:利用协方差定义

考虑误差 \(e = y - \mathbb{E}[y \mid x] = y - \mu_y - K(x - \mu_x)\)

在最优线性估计(即 MMSE 估计)下,\textbf{误差 \(e\) 与观测 \(x\) 不相关}:
\[
\text{Cov}(e, x) = 0
\]

计算:
\[
\text{Cov}(e, x) = \text{Cov}(y - \mu_y - K(x - \mu_x),\ x - \mu_x)
= \text{Cov}(y, x) - K \text{Cov}(x, x)
= \Sigma_{yx} - K \Sigma_{xx}
\]

令其为零:
\[
\Sigma_{yx} - K \Sigma_{xx} = 0 \quad \Rightarrow \quad K = \Sigma_{yx} \Sigma_{xx}^{-1}
\]

因此:
\[
\boxed{ \mathbb{E}[y \mid x] = \mu_y + \Sigma_{yx} \Sigma_{xx}^{-1} (x - \mu_x) }
\]

\end{proof}

\begin{tcolorbox}
式中:

\(\Sigma_{yx}\):\(y\) 和 \(x\) 的协方差(“它们如何一起变化”)\\
\(\Sigma_{xx}^{-1}\):对 \(x\) 的“归一化”(考虑 \(x\) 自身的变化尺度)\\
\(x - \mu_x\):当前 \(x\) 偏离均值的程度\\
整体:根据 \(x\) 的偏差,按协方差比例调整 \(y\) 的预测

这类似于线性回归:  
\[
\hat{y} = \mu_y + \underbrace{(\text{Cov}(y,x) \text{Var}(x)^{-1})}_{\text{回归系数}} (x - \mu_x)
\]

在多元情况下,协方差和方差推广为矩阵. 

\end{tcolorbox}

\printbibliography%[heading=bibliography,title=参考文献]
\end{document}